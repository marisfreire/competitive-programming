\documentclass{article}
\begin{document}

\section*{Combinatorics}

\subsection*{General}

\[
\sum_{0 \leq k \leq n} \binom{n-k}{k} = Fib_{n+1}
\]

\[
\binom{n}{k} = \binom{n}{n-k}
\]

\[
\binom{n}{k} + \binom{n}{k+1} = \binom{n+1}{k+1}
\]

\[
k \binom{n}{k} = n \binom{n-1}{k-1}
\]

\[
\binom{n}{k} = \frac{n}{k} \binom{n-1}{k-1}
\]

\[
\sum_{i=0}^n \binom{n}{i} = 2^n
\]

\[
\sum_{i \geq 0} \binom{n}{2i} = 2^{n-1}
\]

\[
\sum_{i \geq 0} \binom{n}{2i+1} = 2^{n-1}
\]

\[
\sum_{i=0}^k (-1)^i \binom{n}{i} = (-1)^k \binom{n-1}{k}
\]

\[
\sum_{i=0}^k \binom{n+i}{i} = \sum_{i=0}^k \binom{n+i}{n} = \binom{n+k+1}{k}
\]

\[
1\binom{n}{1} + 2\binom{n}{2} + 3\binom{n}{3} + \dots + n\binom{n}{n} = n2^{n-1}
\]

\[
1^2 \binom{n}{1} + 2^2 \binom{n}{2} + 3^2 \binom{n}{3} + \dots + n^2 \binom{n}{n} = (n+n^2)2^{n-2}
\]


\subsection{Vandermonde’s Identity:}
\[
\sum_{k=0}^r \binom{m}{k}\binom{n}{r-k} = \binom{m+n}{r}
\]


\subsection{Hockey-Stick Identity:}
\[
n,r \in \mathbb{N}, \; n>r, \quad \sum_{i=r}^n \binom{i}{r} = \binom{n+1}{r+1}
\]

\[
\sum_{i=0}^k \binom{k}{i}2^i = \binom{2k}{k}
\]

\[
\sum_{k=0}^n \binom{n}{k}\binom{n}{n-k} = \binom{2n}{n}
\]

\[
\sum_{k=q}^n \binom{n}{k}\binom{k}{q} = 2^{n-q}\binom{n}{q}
\]

\[
\sum_{i=0}^n k^i \binom{n}{i} = (k+1)^n
\]

\[
\sum_{i=0}^n \binom{2n}{i} = 2^{2n-1} + \tfrac{1}{2}\binom{2n}{n}
\]

\[
\sum_{i=1}^n \binom{n}{i}\binom{n-1}{i-1} = \binom{2n-1}{n-1}
\]

\[
\sum_{i=0}^n \binom{2n}{i}^2 = \tfrac{1}{2}\Big( \binom{4n}{2n} + \binom{2n}{n}^2 \Big)
\]


\subsection{Highest Power of 2 that divides $\binom{2n}{n}$:}  
Let $x$ be the number of 1s in the binary representation. Then the number of odd terms will be $2^x$. Let it form a sequence. The $n$-th value in the sequence (starting from $n=0$) gives the highest power of 2 that divides $\binom{2n}{n}$.


\subsection*{Pascal Triangle}
In a row $p$, where $p$ is a prime number, all the terms in that row except the 1s are multiples of $p$. \textbf{Parity:} To count odd terms in row $n$, convert $n$ to binary. Let $x$ be the number of 1s in the binary representation. Then the number of odd terms will be $2^x$. Every entry in row $2^n-1, \; n\geq0$, is odd.
An integer $n\geq 2$ is prime if and only if all intermediate binomial coefficients are inserted.
\[
\binom{n}{1}, \binom{n}{2}, \dots, \binom{n}{n-1}
\] are divisible by $n$.

\subsection{Kummer’s Theorem} 
For given integers $n \geq m \geq 0$ and a prime number $p$, the largest power of $p$ dividing $\binom{n}{m}$ is equal to the number of carries when $m$ is added to $n-m$ in base $p$. For implementation, take inspiration from Lucas theorem.

\subsection{Counting Problems}
Number of different binary sequences of length $n$ such that no two 0’s are adjacent:
\[
Fib_{n+1}
\]

\subsection{Combination with repetition}
Choosing $k$ elements from an $n$-element set, order does not matter, repetition allowed:
\[
\binom{n+k-1}{k}
\]


Number of ways to divide $n$ persons in $\tfrac{n}{k}$ equal groups of size $k$:
\[
\frac{n!}{k!^{n/k}(n/k)!} = \prod_{n \geq k} \binom{n-1}{k-1}
\]

Number of non-negative solutions of equation:
\[
x_1+x_2+x_3+\dots+x_k=n \quad \Rightarrow \quad \binom{n+k-1}{n}
\]

Number of ways to choose $n$ ids from $1$ to $b$ such that every id has distance at least $k$:
\[
\binom{b-(n-1)(k-1)}{n}
\]


\subsection*{Restricted Cycle Permutations}

Let $T(n,k)$ be the number of permutations of size $n$ for which all cycles have length $\leq k$:

\[
T(n,k) = 
\begin{cases} 
n! & n \leq k \\
n \cdot T(n-1,k) - F(n-1,k)\cdot T(n-k-1,k) & n > k
\end{cases}
\]

where
\[
F(n,k) = n(n-1)\dots(n-k+1)
\]


\subsection*{Lucas Theorem}

If $p$ is prime, then
\[
\binom{p}{a}{k} \equiv 0 \pmod{p}
\]

For non-negative integers $m$ and $n$ and a prime $p$:
\[
\binom{m}{n} \equiv \prod_{i=0}^k \binom{m_i}{n_i} \pmod{p}
\]
where
\[
m = m_k p^k + m_{k-1} p^{k-1} + \dots + m_1 p + m_0,
\]
\[
n = n_k p^k + n_{k-1} p^{k-1} + \dots + n_1 p + n_0
\]
are the base-$p$ expansions of $m$ and $n$.  
Convention: $\binom{m}{n}=0$ when $m<n$.

\end{document}